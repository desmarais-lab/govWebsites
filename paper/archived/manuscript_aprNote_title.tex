\documentclass[11pt]{article}

%==============Packages & Commands==============
\usepackage{graphicx}
\usepackage{fancyvrb}
\usepackage{tikz}
%%%<
\usepackage{verbatim}
%\usepackage[active,tightpage]{preview}
%\PreviewEnvironment{tikzpicture} 
%\setlength\PreviewBorder{5pt}%

\usepackage{geometry}                        % See geometry.pdf to learn the layout options. There are lots.
% \geometry{a4paper}                           % ... or a4paper or a5paper or ...
%\geometry{landscape}                        % Activat\usetikzlibrary{arrows}e for for rotated page geometry
%\usepackage[parfill]{parskip}            % Activate to begin paragraphs with an empty line rather than an indent
\usepackage{graphicx}                % Use pdf, png, jpg, or eps§ with pdflatex; use eps in DVI mode
                                % TeX will automatically convert eps --> pdf in pdflatex
\usepackage{amssymb}

\usepackage[ruled,vlined]{algorithm2e}
\usetikzlibrary{arrows}
\usepackage{alltt}
\usepackage[T1]{fontenc}
\usepackage[utf8]{inputenc}
\usepackage{indentfirst}
\usepackage{natbib} % For references
\bibpunct{(}{)}{;}{a}{}{,} % Reference punctuation
\usepackage{changepage}
\usepackage{setspace}
\usepackage{booktabs} % For tables
\usepackage{rotating} % For sideways tables/figures
\usepackage{amsmath}
\usepackage{multirow}
\usepackage{color}
\usepackage{dcolumn}
\usepackage{comment}
\usepackage{pgf}
\usepackage{xcolor, colortbl}
\usepackage{array}

\def\mybar#1{%%
    #1 & {\color{red}\pgfmathsetlengthmacro\x{#1*0.006mm}\rule{\x}{4pt}}}

%\pgfmathsetlengthmacro\x{#1^0.1mm}
%\show\x -> 25.0pt

%\usepackage{fullwidth}
\newcolumntype{d}[1]{D{.}{\cdot}{#1}}
\newcolumntype{.}{D{.}{.}{-1}}
\newcolumntype{3}{D{.}{.}{3}}
\newcolumntype{4}{D{.}{.}{4}}
\newcolumntype{5}{D{.}{.}{5}}
\usepackage{float}
\usepackage[hyphens]{url}
%\usepackage[margin = 1.25in]{geometry}
%\usepackage[nolists,figuresfirst]{endfloat} % Figures and tables at the end
\usepackage{subfig}
\captionsetup[subfloat]{position = top, font = normalsize} % For sub-figure captions
\usepackage{fancyhdr}
%\makeatletter
%\def\url@leostyle{%
%  \@ifundefined{selectfont}{\def\UrlFont{\sf}}{\def\UrlFont{\small\ttfamily}}}
%\makeatother
%% Now actually use the newly defined style.
\urlstyle{same}
\usepackage{times}

\usepackage{lscape}
% \usepackage{mathptmx}
%\usepackage[colorlinks = true,
%                        bookmarksopen = true,
%                        pagebackref = true,
%                        linkcolor = black,
%                        citecolor = black,
%                     urlcolor = black]{hyperref}
%\usepackage[all]{hypcap}
%\urlstyle{same}
\newcommand{\fnote}[1]{\footnote{\normalsize{#1}}} % 12 pt, double spaced footnotes
\def\citeapos#1{\citeauthor{#1}'s (\citeyear{#1})}
\def\citeaposs#1{\citeauthor{#1}' (\citeyear{#1})}
\newcommand{\bm}[1]{\boldsymbol{#1}} %makes bold math symbols easier
\newcommand{\R}{\textsf{R}\space} %R in textsf font
\newcommand{\netinf}{\texttt{NetInf}\space} %R in textsf font
\newcommand{\iid}{i.i.d} %shorthand for iid
\newcommand{\cites}{{\bf \textcolor{red}{CITES}}} %shorthand for iid
%\usepackage[compact]{titlesec}
%\titlespacing{\section}{0pt}{*0}{*0}
%\titlespacing{\subsection}{0pt}{*0}{*0}
%\titlespacing{\subsubsection}{0pt}{*0}{*0}
%\setlength{\parskip}{0pt}
%\setlength{\parsep}{0pt}
%\setlength{\bibsep}{2pt}
%\renewcommand{\headrulewidth}{0pt}

%\renewcommand{\figureplace}{ % This places [Insert Table X here] and [Insert Figure Y here] in the text
%\begin{center}
%[Insert \figurename~\thepostfig\ here]
%\end{center}}
%\renewcommand{\tableplace}{%
%\begin{center}
%[Insert \tablename~\theposttbl\ here]
%\end{center}}

\newcommand\independent{\protect\mathpalette{\protect\independenT}{\perp}}
\def\independenT#1#2{\mathrel{\rlap{$#1#2$}\mkern2mu{#1#2}}}
\newcommand{\N}{\mathcal{N}}
\newcommand{\Y}{\bm{\mathcal{Y}}}
\newcommand{\bZ}{\bm{Z}}

\usepackage[colorlinks = TRUE, urlcolor = black, linkcolor = black, citecolor = black, pdfstartview = FitV]{hyperref}


%============Article Title, Authors==================
\title{\vspace{-2cm} Government Websites As Data: \\ Understanding how Mayoral Partisanship Shapes Municipal Website Content
\footnote{This work was supported by the National Science Foundation [1320219, 1637089, 1641047].}}


\author{ Markus Neumann\footnote{Department of Political Science, The Pennsylvania State University, University Park, PA 16802, USA. Email: mvn5218@psu.edu. Corresponding author.} \and Fridolin Linder\footnote{Department of Political Science, Social Media and Political Participation Lab, New York University, New York, NY 10012, USA. Email: fridolin.linder@nyu.edu} \and Bruce Desmarais\footnote{Department of Political Science, The Pennsylvania State University, University Park, PA 16802, USA. Email: bdesmarais@psu.edu. }} \date{\today}

%===================Startup=======================
\begin{document}
\maketitle 



%=============Abstract & Keywords==================

\begin{abstract}

A local government's website is an important source of information about policy priorities,  procedures, and debates. Existing research on government websites has relied on manual methods of website content collection and processing, limiting the scale and scope of website content analysis. In this research note, we propose that the automated collection of website content from large samples of government websites can can compliment more targeted manual methods, and offer contributions through comparative analyses.  We also provide software to ease the use of this data collection method. In our application, we collect a new and innovative dataset---the websites of over two hundred municipal governments in the United States---to study how website content is associated with mayoral partisanship. Using topic modeling methods, we find that cities with Democratic mayors provide more information on policy deliberation and crime control, whereas Republicans prioritize basic utilities and services such as water, electricity, and fire safety. 
%\noindent We explore the effect of transitions of power in municipal governments on the content of their websites. We hypothesize that when party control changes, city administrators modify the contents of their websites in order to fit the agenda of the new incumbent. To test this theory, we study cities in Indiana and Louisiana, two states in which all municipal elections are partisan and the parties of the candidates appear on the ballots. Snapshots of websites before and after transitions of power are acquired through the Wayback Machine. We apply statistical topic models based in latent dirichlet allocation, focusing on changes to the websites. We present results on both which topics see the greatest degree of change associated with transitions in city administrations, and how the topics modified differ with regard to political parties.


\end{abstract}
\thispagestyle{empty}
% \doublespacing
% Description of the possible challenges

\end{document}



